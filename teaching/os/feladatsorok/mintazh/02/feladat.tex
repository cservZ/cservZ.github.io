\documentclass[a4paper,12pt]{article}
\usepackage[magyar]{babel}
\usepackage{t1enc}
\usepackage[T1]{fontenc}
\usepackage[utf8]{inputenc}
\usepackage{caption}
\usepackage{hyperref}
\usepackage{graphicx}
\usepackage{float}
\usepackage{listings}
\usepackage{xcolor}
\usepackage{geometry}
\usepackage{url}
\usepackage{alltt}
\setlength\parindent{0pt}
\setlength{\parskip}{1em}

\geometry{
    a4paper,
    left=20mm,
    right=20mm,
    top=20mm,
}

\begin{document}
\title{\vspace{-22mm}Operációs rendszerek gyakorlat}
\author{}
\date{\vspace{-2mm}2. zárthelyi dolgozat - Minta}

\maketitle

\section*{Általános tudnivalók}
\begin{itemize}
    \item A feladatok megoldására 50 perc áll rendelkezésre.
    \item A megoldásokat az \texttt{1.awk}, \texttt{2.awk}, illetve \texttt{3.awk} fájlokba kell elkészíteni.
    \item A megoldást tartalmazó fájlokat helyezd egy \texttt{VEZETEKNEV\_KERESZTNEV} nevű mappába (értelemszerűen a saját adataid behelyettesítésével)! A mappát egyetlen tömörített ZIP fájlként töltsd fel a CooSpace rendszerébe \texttt{VEZETEKNEV\_KERESZTNEV.zip} néven!
\end{itemize}

\section*{1. feladat: Fájlméret összegző (6 pont)}
Írj AWK szkriptet \texttt{1.awk} néven, amely egy olyan fájlt kap paraméterben, amelynek minden sora két, pontosvesszővel elválasztott adatot tartalmaz: rendre egy fájl nevét és méretét! A szkript keresse meg a paraméterben kapott fájlban azokat a sorokat, amelyekben a fájlnév a következőképpen épül fel:
\begin{itemize}
    \item A fájlnév elején pontosan 4 darab kisbetű szerepel.
    \item Ezután legalább 1 darab számjegy következik.
    \item Rögtön ezután a \texttt{.png} vagy \texttt{.jpg} kiterjesztések valamelyike szerepel. A fájl neve pontosan itt ér véget, tehát az ennél bővebb fájlnevek nem fogadhatók el.
\end{itemize}

Írasd ki a fenti feltételeknek megfelelő fájlok neveit a konzolra! Írasd ki azt is, hogy a megtalált fájlok mérete összesen mekkora!

\textbf{Példa:}

\begin{alltt}
> ./1.awk 01/files.csv
dogs2.jpg 
pepe123456.png 
meme2021.jpg 
yeet55.jpg 
----------------------------------------------- 
A megtalalt fajlok merete osszesen 409374 bajt.
\end{alltt}

\section*{2. feladat: Leghosszabb szó (6 pont)}

Írj AWK szkriptet \texttt{2.awk} néven, amely egy olyan fájlt kap paraméterben, amely egy több soros szöveget tartalmaz! A szkript írja ki a konzolra a bemeneti fájlban található leghosszabb szót \textbf{csupa nagybetűkkel}! A szövegben található írásjelekkel nem kell foglalkoznod, ezeket is nyugodtan számold bele a szavak hosszába!

\textbf{Példa:}
\begin{alltt}
> ./2.awk 02/post.txt
HALLGATOTARSAIET
\end{alltt}

\section*{3. feladat: Jelesek száma (8 pont)}

Írj AWK szkriptet \texttt{3.awk} néven, amely egy olyan fájlt kap paraméterül, ami az Operációs rendszerek gyakorlaton elért hallgatói eredményeket tartalmazza! Minden sorban 5 darab, pontosvesszővel elválasztott adat található: rendre a hallgató neve, szakja, az 1. ZH pontszáma, a 2. ZH pontszáma és a plusz pontok száma. A szkript határozza meg, hogy a különféle szakokon hány jeles született! \textbf{A feladat megoldása során használj tömböt} (tehát ne csak a példafájlban szereplő szaknevek esetén működjön jól a szkript)!

Egy hallgató akkor kap jelest a gyakorlaton, ha a ZH-kon elért pontoknak és a plusz pontoknak összege \textbf{legalább 35 pont}. A fájl első sora egy fejlécsor, ezt a program ne dolgozza fel!

\textbf{Példa:}
\begin{alltt}
> ./3.awk 03/points.csv
Proginfo szakon 3 hallgato kapott jelest.
Uzemmernok szakon 1 hallgato kapott jelest. 
Gazdinfo szakon 2 hallgato kapott jelest. 
Mernokinfo szakon 2 hallgato kapott jelest.
\end{alltt}

\textbf{Jó munkát!}
\end{document}