\documentclass[a4paper,12pt]{article}
\usepackage[magyar]{babel}
\usepackage{t1enc}
\usepackage[T1]{fontenc}
\usepackage[utf8]{inputenc}
\usepackage{caption}
\usepackage{hyperref}
\usepackage{graphicx}
\usepackage{float}
\usepackage{listings}
\usepackage{xcolor}
\usepackage{geometry}
\usepackage{url}
\usepackage{alltt}
\setlength\parindent{0pt}
\setlength{\parskip}{1em}

\geometry{
    a4paper,
    left=20mm,
    right=20mm,
    top=20mm,
}

\begin{document}
\title{\vspace{-22mm}Operációs rendszerek gyakorlat}
\author{}
\date{\vspace{-2mm}1. zárthelyi dolgozat - Minta}

\maketitle

\section*{Általános tudnivalók}
\begin{itemize}
    \item A feladatok megoldására 50 perc áll rendelkezésre.
    \item A megoldásokat az \texttt{1.sh}, \texttt{2.sh}, illetve \texttt{3.sh} fájlokba kell elkészíteni.
    \item A megoldást tartalmazó fájlokat helyezd egy \texttt{VEZETEKNEV\_KERESZTNEV} nevű mappába (értelemszerűen a saját adataid behelyettesítésével)! A mappát egyetlen tömörített ZIP fájlként töltsd fel a CooSpace rendszerébe \texttt{VEZETEKNEV\_KERESZTNEV.zip} néven!
\end{itemize}

\section*{1. feladat: Zümi (6 pont)}
A \texttt{users.csv} fájl Discord felhasználók adatait tartalmazza. Az egyes sorokban szereplő, pontosvesszőkkel elválasztott adatok: a felhasználói azonosító (\texttt{felhasználónév\#tagcímke} formátumban), a felhasználó szakja és állapota.

Írj csővezetéket egy \texttt{1.sh} nevű állományba, amely a parancssori paraméterben érkező fájlból kiírja az \texttt{uzemmernok} szakos hallgatók felhasználónevét ábécé sorrendben (a tagcímke nélkül)! Feltehetjük, hogy a bemeneti fájlban az \texttt{uzemmernok} szöveg csak a szakok között fordul elő.

\textbf{Példa:}

\begin{alltt}
> cat 01/users.csv
laszti#4121;proginfo;online
float#0466;mernokinfo;offline
zum_zum#1234;uzemmernok;offline
the_serbian_lemon#2773;proginfo;online
egermutato57#6969;gazdinfo;offline
szte2k1#9572;mernokinfo;online
happyzumi#1024;uzemmernok;online

> ./1.sh 01/users.csv
happyzumi
zum_zum
\end{alltt}

\section*{2. feladat: Abszolútérték-összeg (6 pont)}
Írj BASH szkriptet \texttt{2.sh} néven, amely kiírja a konzolra a parancssori paraméterben kapott számok abszolútértékének összegét (tehát először venni kell minden számnak az abszolútértékét, és ezeket az abszolútértékeket kell összeadni)! Hibakezeléssel nem kell foglalkoznod, feltesszük, hogy a program paraméterezése helyes.

\textbf{Példa:}

\begin{alltt}
> ./2.sh 7 -4 -1 0 2 -5 8 3
A parameterek abszolutertekenek osszege: 30
\end{alltt}

\section*{3. feladat: Az ember legjobb barátja (8 pont)}
Írj BASH szkriptet \texttt{3.sh} néven, amely beleírja egy \texttt{out.dat} fájlba a paraméterben kapott mappa összes olyan \texttt{txt} állományának tartalmát, amely tartalmazza a \texttt{kutya} szöveget!

Még a program elején vizsgáld meg, hogy a kapott paraméter mappa-e! Abban az esetben, ha nem mappa, akkor lépjen ki a program 42-es hibakóddal!

\textbf{Példa:}

\begin{alltt}
> ls 03
a.txt   b.txt   c.php   d.txt

> ./3.sh 03

> cat out.dat
Ez egy olyan txt kiterjesztesu fajl, amelyben 
szerepel a kutya szoveg. Emiatt ennek a fajlnak 
a tartalmat bele kell irni a kimeneti allomanyba.
"Egy kutya tiz evvel meghosszabbithatja az eletunket. 
A szeretet, a gyengedseg, a torodes es a barati kapcsolat, 
amit egy kutya jelent, mersekli a stressz negativ hatasait."
- David R. Hawkins
\end{alltt}

\textbf{Jó munkát!}
\end{document}
